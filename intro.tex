\chapter{Introduction}

Every year, fighting forest fires costs taxpayers in the United States millions of dollars. From 2002 to 2012, the average amount spent per year on forest fire suppression by the federal government was \$962 million, which only amounted to 32\% of the entire federal wildfire protection funds \cite{forestcost}. This cost only covers the federal funds that are spent on fighting and preventing forest fires. It does not include the individual and environmental cost of loss of property and habitat. The highest cost that occurs during the efforts to fight a forest fire is the loss of life incurred by fire fighters. The ability to better predict the behavior of a wildfire greatly increases the effectiveness of fire fighting efforts, thereby reducing all the costs incurred during a forest fire. Another application domain for forest fires is the training of fire fighters. A sample forest fire could be provided in training scenarios which would allow fire fighters to have as close to hands-on forest fire training before they are exposed to real wildfires. 

In order to simulate wildfires, scientists have developed several methods for modeling the propagation of fire \cite{roth,BEHAVE,1983roth}. These fire models are based on the properties of the environment in which the forest fire takes place. Such properties include, but are not limited to, fuel load, fuel type, wind, live moisture, dead moisture, and crown height. Incorporating these and other variables into spread models can allow for accurate prediction of where a fire will spread and how quickly it will arrive. Research into developing fuel and moisture models is active to this day. These models provide the basis for the properties on which forest fire simulation is based.

The ability to realistically simulate forest fires is desirable because it allows fire experts to more accurately predict the impact of their fire-fighting decisions. Possible manipulations to the wildfire environment include adjusting moisture content to simulate a water drop, adjusting fuel loads where a simulated bulldozed treeline could exist, or reverse spread testing in which a fire started by firefighters would burn the fuel away from the advancing wildfire. Unfortunately, the amount of data required for realistic fire simulations requires a large amount of computation time to produce an accurate simulation. The more accurate and fine-grained the simulation, the longer it takes to process the data. A forest fire is a dynamic entity, therefore the ability of a simulator to run in real time is necessary for it to be an effective tool. The more complex and accurate a simulator is, the more useful it is to fire scientists. There are multiple aspects to accurately modeling the spread of a forest fire. The main four fire properties which influence the spread of a fire are base fires, crown fires, fire acceleration, and spotting \cite{firereview}. 

Using the GPU as a general purpose computing device has become popular in recent years, especially on problems which require a large amount of data processing. The GPU is ideally suited to high volume data processing applications because it can processes millions of inputs simultaneously, while a CPU may only process up to a few at a time \cite{cudabyexample}. This work has developed a fire simulation library which allows a user to run base propagation with fire acceleration tests on real-world data, as well as testing the crowning conditions and a simple spotting method. 

The remainder of this paper is structured as follows. Chapter \ref{chapter:background} contains the background information on forest fire models and the existing forest fire simulators. Chapter \ref{chapter:gpuSim} outlines of the work accomplished in this paper. Chapter \ref{chapter:design} presents the main components of the library's software specifications and design. Chapter \ref{chapter:implementation} describes the implementation details of the forest fire simulation library. Chapter \ref{chapter:results} presents the results of the timing tests and fire simulations performed for this work. Finally, Chapter \ref{chapter:conclusions} presents ideas for future enhancements for the fire library. 