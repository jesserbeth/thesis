\chapter{Real-time GPU-based Wildfire Simulation}
\label{chapter:gpuSim}
I NEED A CATCHY NAME FOR MY SIMULATOR. 

XXX is a wildfire simulation library that allows the user to take advantage of the highly-parallel nature of the GPU. The propagation, crowning, and spotting are all implemented both sequentially and as kernels on the GPU using the programming language CUDA \cite{cuda}. The novelty of this work exists in the comprehensive implementation of a forest fire simulation library that can leverage the highly-parallel nature of the GPU. 

Although several fire simulators have been implemented in the past, including BEHAVE, FireLib, and FARSITE, they do not use the GPU to compute the fire spreads \cite{fireLib}\cite{BEHAVE}\cite{FARSITE}. Because of their sequential nature, as well as the high demands on amount of data needed for an accurate simulation, the time required to run these simulations make them unsuitable for real-time applications such as training or live wildfire prediction. The dynamic nature of a wildfire makes the need to run in real-time not only helpful, but necessary. In order to make a simulator operate in real-time, a tradeoff exists between accuracy and processing time. It is not cost-effective to expect fire fighters to have access to multi-CPU processing platforms, but the cost of a single GPU is not unreasonable to expect for a wildfire expert to obtain. For these reasons, and reasons of ease of use given the programming language CUDA, wildfire simulation on the GPU is the focus of this work. The only previous work to port the computation load of propagation simulation to the GPU was vFire \cite{vFire}, which was implemented before programming for the GPU became easily accessible. 

The nature of wildfire simulation begins with the need to step through time to give an accurate estimation of the behavior of the wildfire. At each time step, computation must be done on every cell of the fire. At each time step the spread of the fire, the acceleration of the fire, the crowning test, and the spotting check all need to be calculated. Leveraging the GPU to operate as the computational workhorse is ideal because every cell needs the same processing done on it, no matter what the environment. The GPU can process each cell in parallel to its neighbors. 

This parallel computation allows each time step to be processed as one pass on the GPU. While the cores on the GPU are not as powerful as CPU cores, the real benefit to this computation style comes when many passes over the data are needed. Passing data between the GPU and CPU is a bottleneck as far as speed is concerned. This is why running all the computations on the GPU is ideal for the simulator. At a fine granularity, the benefits of speed in processing time outweigh the negatives of having to pass data back and forth between the CPU and GPU.

The focus of this work is improving the runtime needed for the simulation of a wildfire by using the GPU as the processing workhorse. The model developed is based upon the existing work found in many simulators including vFire and FARSITE. This work moves beyond the implementation of the work found in vFire, which can be thought of as the parent research to this implementation. Not only does this work reimplement the work of vFire on the modern platform for GPU computing by using CUDA, but adds additional functionality in its three propagation methods and the implementation of a basic spotting model. 

