\chapter{Modeling and Implementation}
\label{chapter:implementation}

\section{Overview}
Once the preprocessing is complete, one of three simulation methods was applied: Minimal Time, Iterative Minimal Time or Burn Distances. Since the goal of a forest fire simulator is to operate in real-time, it is unrealistic to expect to wait that long to receive the simulation data. After thirty minutes, the fire will have changed enough for the simulation's predictions to be irrelevant. The following algorithm outlines the simulation steps from start to finish:

\begin{algorithm}
  \caption{Simulation Progression}
  \label{alg:sim}
%   \begin{algorithmic}[1] <- this will give you line numb
  \begin{algorithmic}
  \STATE InitializeTerrainData();
  \STATE CalculateSpreadRates();
  \WHILE{Simulation !Complete}
  \STATE RunPropagationSimulation();
  \ENDWHILE  
  \STATE GenerateOutputFile();
  \end{algorithmic}
\end{algorithm}

The InitializeTerrainData() and CalculateSpreadRates() functions in Algorithm \ref{alg:sim} are part of the preprocessing in this project. These are not addressed in detail in this paper due to space constraints. The RunPropagationSimulation() portion is the focus of this paper and will be outlined in more detail in the following sections.

In order to address the runtime analysis needed for this paper, a sequential version of the kernels was also implemented. The equations and spread methodologies are the same between the kernels and the sequential implementation. However, many of the data concurrency problems were not a problem in the sequential version, and therefore some of the secondary kernels which will be addressed were not implemented in the sequential versions. The sequential implementations are not discussed in detail in this paper because they are not the main focus of the library and exist mostly for comparison purposes.  


\section{Surface Fire}
There are several potential approaches to calculating the propagation of a wildfire. This paper implemented three methods for iterating through a simulation to calculate the time of arrival map for a simple one-source forest fire under constant terrain and wind conditions. The first two spread methods (Minimal Time and Iterative Minimal Time) are based on stepping through time independent of specific fuel conditions and are based on the paper by Sousa, dos Reis, and Pereira \cite{gpufire}. The third spread method implemented in this paper (Burn Distances) was based on code and methods found in vFire \cite{vFire}. vFire implemented an accurate spread rate calculator based on Rothermel's fire spread equations and the fire spread and fuel model data to propagate based upon the physical burning of fuel.

The model for propagation rate used in this paper was the same for all three spread methods. The model is based on Rothermel's fire spread equations, and is found increately Equation \ref{eq:rate}. This equation was derived by the creators of vFire \cite{vFire}, and the data processing done in the preprocessing phase of this project was based on their work. The preprocessing step translates the equation from Rothermel's work seen in Equation \ref{eq:roth} to what is seen in the following Equation \ref{eq:rate}. The preprocessing data is out of the scope of this paper and is not covered in detail. 

\begin{equation}
r(\Theta ) = R_{max}\frac{1.0 - \varepsilon }{1.0 - \varepsilon cos(\phi-\Theta )}
\label{eq:rate}
\end{equation}

$R_{max}$ is the maximum rate at which a fire can spread. $\varepsilon$ is the eccentricity of the fire, which is based on wind and slope data. $\phi$ is the orientation. $\Theta$ is the direction in which the fire is spreading. $R_{max}$, $\varepsilon$, and $\phi$ are all computed based on the terrain data before the propagation simulation takes place. This is done in the preprocessing stage because the rates at each cell do not change until the forest model changes. An interactive simulator could potentially allow for these variables to change (i.e. modeling a water dump from a helicopter or a bulldozer tearing down a line of trees) but that is outside the scope of this research paper. To implement these features, changes would be made to the fuel and moisture models used to determine the possible rate of change in a cell. The direction $\Theta$ is computed based on which neighbor is being examined at the time. 

During the preprocessing phase, there are several data files which need to be processed and interpolated to be of the same size. The fuel data and slope data are stored in files containing interpolation data such as size of cell, width, and height of the data grid. The Geospatial Data Abstraction Library (GDAL) was used to interpolate the data from the terrain and fuel files into the desired size of simulation \cite{GDAL}. Wind data is incorporated into the spread rate calculations as a 2D vector for each cell in the grid. The wind data contains a direction and magnitude. The fuel models provide the detailed parameters by which the rate of of spread is calculated. There are potential areas in this processing phase (such as calculating the Rothermel spread properties) that could be parallelized to improve overall running times of this simulator. However, the focus of this paper was to explore the potential for calculating fire spread on the GPU, so these possibilities were not addressed and are left for future work. 


\section{Fire Propagation}
\subsection{Overview}
In a wildfire spread simulation, there has to be a way to iterate through time and check for the propagation of the fire. This thesis implemented three such spread methods, which provide different fireline shapes as they move through the simulation. Each of the propagation methods are written as CUDA kernels \cite{cuda}. The implementation details of each unique method follow in this section of the paper. The following figure, Figure \label{fig:spreadTypes_2}, is a repeat of Figure \ref{fig:spreadTypes} for ease of reference to the reader. It depicts the neighbor access methods for the propagation methods.

\begin{figure}%[!t]
\centering
\includegraphics[width=\linewidth]{figures/background/spread_methods.PNG}
\caption{Neighbor access methodology for each of the propagation methods. From left to right: Minimal Time, Iterative Minimal Time, Burn Distances}
\label{fig:spreadTypes_2}
\end{figure}

\subsection{Burn Distances}
Burn distances is a kernel which simulates the burning of the physical distance between fire cells. In order to limit unnecessary computations, cells that are already on fire are skipped in the kernel. Only those that are not on fire check their neighbors to see if during this time step they are set alight.The pseudocode for the algorithm for the BD propagation method is found in Algorithm \ref{alg:BD}.

\begin{algorithm}[H]
  \caption{Burn Distances Algorithm}
  \label{alg:BD}
  \begin{algorithmic}
  \FOR{$cell = 0$ to $numCells$}
  \STATE $//$ Check to see not on fire
  \IF{$ignTime[cell] == INF$}
  \STATE $Skip$
  \ENDIF
  \STATE $//$ Check Neighbors for spread
  \FOR{$n = 0$ to $7$}
  \IF{$ignTime[neighborCell] < INF$}
  \STATE $Skip$
  \ENDIF
  \STATE $ROS$ = Compute ROS according to Equation \ref{eq:rate}
  \STATE $burnDistance$(totDist[neighborCell], ROS, timeStep)
  \IF{distance is burnt}
  \STATE $ignTime[neighborCell] = timeNow$
  \ENDIF
  \ENDFOR
  \ENDFOR
  \end{algorithmic}
\end{algorithm}

During each kernel call, each cell is processed independent of its neighbors. However, when a new cell ignites, the data must be written out to the vector somehow. There is no way to ensure which threads are reading and writing all at the same time. In fact, it was found that race conditions existed when the threads were reading from and writing to the same vector. In order to fix this problem, an input and output vector were created. Threads read only from the input vector and write to the output vector. A secondary kernel was written to copy the data values from one vector to another between the propagation kernel calls. The algorithm for the copy kernel may be found in Algorithm \ref{alg:BD_copy}:

\begin{algorithm}[H]
  \caption{Burn Distances Algorithm}
  \label{alg:BD_copy}
  \begin{algorithmic}
  \FOR{$cell = 0$ to $numCells$}
  \STATE $//$ Copy from output back to input 
  \STATE $InputVector[cell]$ = $OutputVector[cell]$
  \ENDFOR
  \end{algorithmic}
\end{algorithm}

An issue with this method arose from the static time step and needed to be addressed. Figure \ref{fig:accerr} shows the problem that can arise from the fixed time step. Figure \ref{fig:accerr} (a) shows the initial propagation. In this scenario, imagine that the propagation rate for b is much higher than a, but because the time step is large enough, they both propagate approximately one cell per time step. Figure \ref{fig:accerr} shows the second propagation step in time. In this example, a' propagates to a''. Because of the faster propagation rate of b, b' should propagate to b'' and b'' to b''' before a' propagates to a''. In the case where the time step is too large, the time of arrival would show an erroneous value for the cell holding a''. To fix this issue, the time step in the simulation must be set to a small enough value to avoid this error. The time step should be set to the smallest possible time it takes fire to propagate from one cell to another. 

\begin{figure*}
% \begin{figure}
\centering
%   \includegraphics[width=\textwidth,height=4.2cm]{burn_patterns}
  \includegraphics[width=\linewidth]{figures/implementation/Acceleration_error.png}
  \caption{The possible error for a fixed time-step propagation method. (a) shows the initial step with two lit cells a and b propagating to cells a' and b'. (b) shows a' propagating to a'' in the next time step. (c) is the situation where b'' would have propagated faster to the slot occupied in the previous step by a'', but because of the fixed time step, it would not propagate to an already lit cell.}
  \label{fig:accerr}
% \end{figure}
\end{figure*}

A cell ignites when the distance from an ignited neighbor has been completely burned. Each cell stores the distance that has been consumed by the fire in each direction to its neighbors. The simulation checks a cell's neighbors for ignition, and then uses their properties to burn the amount of distance in that time step towards the current cell as seen in Figure \ref{fig:spreadTypes_2}. The cell ignites when one of its neighbors burns the distance completely. The cell then inherits the properties of the fire that propagated to it. The equation to determine how much distance is burned can be found in Equation \ref{eq:burn}. 

\begin{equation}
d = d - r \Delta t
\label{eq:burn}
\end{equation}

Where $d$ is the distance that needs to be consumed. It is slowly decremented over time by the rate of spread ($r$) times the time step size ($\Delta$ $t$). In order to account for the fact that an 'overburn' could occur with fixed time steps. The exact time of arrival that is calculated is dependent on the exact time of arrival that the fire would have arrived at the cell. The equation used to find the exact time of arrival is found in Equation \ref{eq:overburn}. 

\begin{equation}
TOA = t_{now} + \frac{d_{over}}{r}
\label{eq:overburn}
\end{equation}

Where $TOA$ is the time of arrival that is written out to the output time of arrival map, $t$ is the time in the simulation during which the propagation is occurring, $d_{over}$ is the distance the fire burnt past the desired difference, and $r$ is again the rate of spread. 

\subsection{Minimal Time}

The Minimal Time algorithm was developed using the information provided in Sousa, dos Reis, and Pereira's paper \cite{gpufire}. In this propagation method, each cell spreads outwards towards its neighbors. At each time step, a cell is examined to see if it is on fire. If the cell is burning, then it calculates the propagation time to the surrounding cells. If any of the neighbors are on fire, it is ignored. If the neighbor is unlit, then the time of arrival for that neighbor is calculated. A neighbor which has been lit in the same timestep as the current spread calculations will still be examined for the possibility of a sooner time of arrival from the current cell. A difference between this kernel and the Burn Distances kernel is the time stepping is dynamic. Each time a cell is successfully ignited, the new time of arrival is compared against a 'timeNext' variable to see if it a sooner time of arrival. The next step in time is based on the time in the simulation when the soonest cell ignites. The algorithm for the Minimal Time kernel may be found in Algorithm \ref{alg:MT}.

\begin{algorithm}[H]
%   \small
  \caption{Minimal Time Algorithm}
  \label{alg:MT}
%   \begin{algorithmic}[1] <- this will give you line number
  \begin{algorithmic}
  \FOR{$cell = 0$ to $numCells$}
  \IF{$timeNext > ignTime[cell] AND ignTime[cell] > timeNow$}
  \STATE $timeNext = ignTime[cell]$
  \ELSIF{$ignTime[cell] == timeNow$}
  \STATE $//$ Propagate Fire
  \FOR{$n = 0$ to $15$}
  \STATE $//$If neighbor is unburned
  \IF{$ignTime[neighborCell] > timeNow$}
  \STATE $ROS$ = Compute ROS according to Equation \ref{eq:rate}
  \STATE $ignTimeNew = timeNow + L_n / ROS$
  \IF{$ignTimeNew < ignTime[neighborCell]$}
  \STATE $igntime[neighborCell] = ignTimeNew$
  \ENDIF
  \IF{$ignTimeNew < timeNext$}
  \STATE $timeNext = ignTimeNew$
  \ENDIF
  \ENDIF
  \ENDFOR
  \ENDIF
  \ENDFOR
  \end{algorithmic}
\end{algorithm}

There are a few challenges which present themselves when implementing the algorithm. As in the Burn Distances kernel, data synchronization is an issue. Each cell in the fire is processed by one thread at a time, but every thread need access to the time of arrival map for both reading and writing. However, the solution to the problem between the two kernels is different. In order to minimize these race conditions, atomic operations were used to ensure that data integrity is maintained. CUDA's AtomicMin() operation was used to ensure that a cell that was writing to a data location was not overwriting a smaller time of arrival, which is the correct value that needed to be stored \cite{cuda}. Atomic operations in CUDA are designed to lock read-write access so a single thread has exclusive access rights. The AtomicMin() operation ensures that the minimum of two integers is the one which is stored at the memory location \cite{cudabyexample}. This solves the problem of one thread reading a value for its comparison then another writing a value to that same location, which would mean the value against which the calculations are compared would be inaccurate. A visual example of this read-write problem can be seen in Figure \ref{fig:readwrite}. The reason the AtomicMin() function is a solution to the read-write problem for this kernel as opposed to the Burn Distances kernel is that the Minimal Time Kernel processes integers. AtomicMin() has no implementation for floating point precision data values. 

\begin{figure}%[!b]
\centering
%   \includegraphics[width=\textwidth,height=4.2cm]{burn_patterns}
  \includegraphics[height=4.2cm]{figures/implementation/read-write-issue.png}
  \caption{This is an example of the problem syncing thread read access and write access. There is no way to stop one thread writing to another cell before the value is read by another cell.}
  \label{fig:readwrite}
\end{figure}

A secondary kernel was introduced to manage the time update between time steps in the simulation. The secondary kernel was found to be the most efficient solution to this problem. CUDA does not allocate threads in any specific order, which means thread 1 could be the last to finish calculations while thread 1,000 could be the first \cite{cudabyexample}. In order to step through time in the MT propagation method, the timeNext variable needs to be updated after all calculations finish. Since there is no way to ensure which thread would be the last to finish operating on the data, another method for iterating the variable was needed after all threads had finished their computation. The secondary kernel is called after the first terminates, which ensures that all threads have finished their computation before the timeNext variable is updated. Copying the two-element time vector back to the host device and managing the update there was also tested, but it was found to be faster to write a new kernel to update the data without copying anything back to the host device.

\subsection{Iterative Minimal Time}
The parallel implementation of the IMT kernel is also very similar to the sequential implementation, and for the same reasons as the MT kernel pseudocode is not included, neither is IMT's. The same data synchronization issues are also found in this kernel, where reading from and writing to the same cells would cause race conditions. A simple example of this problem may be seen in Figure \ref{fig:readwrite}. There is no way to stop a thread from writing to an output position before another thread reads in the data it needs to do its own spread calculations. Figure \ref{fig:readwrite} (a) shows the writing of the value from a to a'. The value that existed before a' was the appropriate value for c to read in to do its calculations. Instead as seen in Figure \ref{fig:readwrite} (b), it is the result value from a that it reads into do its calculations. This error causes race conditions to occur and artifacts to appear in the simulation. Instead of using atomic operations to solve this problem, two time of arrival vectors are passed to the kernel at startup. The two kernels act as an input and output kernel, reading from the former and writing to the latter. This introduces a new problem with maintaining accurate spread data across the input and output vectors. The secondary kernel in the IMT method is both used to copy data from the output back into the input as well as checking for the terminating condition for the simulation. 

\begin{algorithm}[H]
%   \small
  \caption{Iterative Minimal Time Algorithm}
  \label{alg:IMT}
  \begin{algorithmic}
  \FOR{$cell = 0$ to $numCells$}
  \STATE $//$ Check for simulation completion:
  \IF{$|ignTime[cell] - ignTimeNext[cell]| < thresh$}
  \STATE $//$Mark as converged
  \ENDIF
  \IF{$ignTime[cell] > 0$}
  \STATE $ignTimeMin = INF$
  \STATE $//$Propagate Fire
  \FOR{$n = 0$ to $15$}
  \STATE $ROS$ = Compute ROS according to Equation \ref{eq:rate}
  \STATE $ignTimeNew = timeNow + L_n / ROS$
  \STATE $ignTimeMin = MIN(ignTimeNew, ignTimeMin)$
  \ENDFOR
  \ENDIF
  \ENDFOR
  \end{algorithmic}
\end{algorithm}

\section{Fire Acceleration}
A fire does not begin burning at its maximum rate of spread. Instead, it accelerates slowly towards its maximum rate over time as it burns. In this simulator library, fire acceleration is based on the work done by vFire \cite{vFire}. vFire in turn used the model presented by FARSITE \cite{FARSITE}. The formula is a simple logarithmic model that state the rate at some time is dependent on the time allowed for the fire to accelerate to its maximum rate depending on the current conditions of the cell in which it burns. This model was developed by the Canadian Forestry Fire Danger Group and Mcalpine and Wakimoto \cite{accel_canada,accel_mcalpine}. 

Given the maximum spread rate $R_{max}$, the maximum spread rate at time $t$ is given by: 

\begin{equation}
R(t) = R_{max}(1.0 - e^{a_at})
\label{eq:R(t)}
\end{equation}

where $a_a$ is the acceleration constant and is drawn from the fuel model appropriate for the current fire cell. The time $t_{max}$ required to achieve the maximum spread rate given the current spread rate $R$ is given by: 

\begin{equation}
t_{max} = \frac{1.0 - \frac{R}{R_{max}}}{a_a}
\end{equation}

To calculate how much the rate should change per time step, the spread rate for every cell is increased at each time step. The rate of increase $dR$ is given by: 

\begin{equation}
dR = \frac{dt}{t_{max}}(R_{max} - R_{current})
\end{equation}

A newly ignited fire will begin with a spread rate of zero and increase steadily to its maximum rate. Once the maximum spread rate is reached, the simulation does not accelerate past it. As the fire spreads to a new cell, that cell inherits the rate properties of the cell which ignited it. If the maximum spread rate of the newly ignited cell is smaller than the rate it inherits, it is clamped to its maximum. If the rate is slower than the maximum spread rate of the newly ignited cell, it is accelerated using the following equation: 

\begin{equation}
dt = timeOfArrival(thisCell) - max(baseTime, timeOfArrival(ignitingCell))
\end{equation}

The time of arrival issue referred to by Figure \ref{fig:accerr} was a problem caused by acceleration in vFire. However, the time stepping methods in the MT and IMT methods do not have this issue as they have dynamic timestepping, and so they are not bound by the same limitations. The BD method's fix was addressed earlier in the chapter. 

\section{Crown Fire}
The occurrence of crown fires is important for both the overall spread of the fire as well as the phenomena of spotting. Torching occurs when the crown fire spreads from the base of the trees to the top of the trees, and is the moment when the possibility of spotting occurs. The crown fire begins as passive and has the ability to move to an active state, which increases the maximum rate of spread by the fire. The implementation presented in this paper is again based on the implementation found in vFire \cite{vFire}. Their method was based entirely on FARSITE's implementation, which uses Van Wagner's methods found in \cite{wagner1977,wagner1993}. 

In this implemenation, the maximum passive crown rate is given by: 

\begin{equation}
R_{max_{crown}} = 3.34R_{10}
\end{equation}

The necessary components needed for calculating the actual maximum spread rate of an active crown fire in addition to the base propagation are all calculated based on properties found in the fuel model for each cell. The threshold which determines the point at which a passive crown fire is promoted to active (RAC) is given by: 

\begin{equation}
RAC = \frac{3.0}{CBD}
\end{equation}

The reaction intensity ($I_b$) of the fire can be obtained by multiplying the current spread rate by the intensity modifier. The intensity modifier is calculated using the same data which goes into the Rothermel equations.

\begin{equation}
I_b = R * IntensityModifier
\end{equation}

$I_o$ is the threshold intensity for a crown fire to occur. The moment when this threshold is passed is the moment when Torching occurs and a flag must be set for the spotting check. $I_o$ is given as follows: 

\begin{equation}
I_o = (0.010CBH(460 + 25.9M))
\end{equation}

Where CBH is the crown base height and M is the foliar moisture content. The crown base height is stored in one of the input files needed to operate this simulator. In this implementation (as well as the one found in vFire), foliar moisture is simply set to be 1.0. In the context of the simulation, the $I_o$ values only need to be calculated once and are included in the preprocessing phase of the simulator. 

In order to determine the actual spread rate of the fire after crowning, there are two parameters which must be calculated: the crown coefficient ($a_o$), and the surface fuel consumption ($R_o$). $R_o$ can be calculated by using the reaction intensity of the fire, the current rate of spread, and the threshold intensity, given by: 

\begin{equation}
R_o = I_o \frac{R}{I_b}
\end{equation}

The crown coefficient $a_c$ may then be calculated as follows:

\begin{equation}
a_c = \frac{-ln(0.1)}{0.9(RAC - R_o)}
\end{equation}

The crown fraction burned (CFB) of the cell determines the actual maximum current spread rate of the crown fire. The CFB is found using the following equation: 

\begin{equation}
CFB = 1 - e^{-a_c(R-R_o)}
\end{equation}

In order to determine the actual maximum rate of spread, the current rate is added to the $CFB$ that is scaled based on how close to the maximum crown rate the current rate is, as given by: 

\begin{equation}
R_{actual} = R + CFB * (R_{max_{crown}} - R)
\end{equation}

The maximum between this new calculated value and the maximum spread rate for the base fire is the actual value that is passed back as the maximum spread rate of a fire cell containing an active crown fire. 

\section{Spotting}
Spotting is the phenomena that occurs when a fire brand is lofted into the air and blown ahead of the fire to potentially start a new fire ahead of the advancing fire wall. A fire brand is a burning piece of wood from a tree or debris found in the burning fire cell \cite{firereview}. The implementation found in this paper is based on what was developed in the FARSITE fire simulator \cite{FARSITE}. FARSITE uses Albini's 1979 model for spotting propagation to determine the location of the potential new fire \cite{albini}. In this implementation, the fire brands are assumed to be cylinders, which is an approximation to simplify the aerodynamics of the model. Figure \ref{fig:spot_diagram} shows the stages of spotting and will be referenced for the rest of this section as the implementation of spotting is described. 
\begin{figure}%[!b]
\centering
  \includegraphics[width=\textwidth]{figures/implementation/spotting_diagram.png}
  \caption{A diagram to represent the phases of spotting, taken from \cite{firebehaveref}.}
  \label{fig:spot_diagram}
\end{figure}

When torching occurs, the spotting calculation can begin. The first important component which needs to be computed is the time it takes for the particle to reach its maximum height $t_t$. $t_t$ is dependent on four time components given by: 
\begin{equation}
t_t = t_0 + t_1 + t_2 + t_3
\end{equation}

Where $t_0$ is the time of steady burning tree crowns. This is a data value that is dependent on the properties of the forest in the cell where spotting is occurring. This value was not available for this work, so a placeholder value was created for the time being and is a constant value for the simulations in the results section of this paper. 

$t_1$ is the time it takes a particle to travel from its point of ignition to the flame tip. This paper makes the assumption that the fire brands originate from the top of the tree crown, but the actual value could be anywhere between the first branches on the tree capable of producing a brand to the tree crown. The equation used to calculate $t_1$ is as follows: 
\begin{equation}
t_1 = 1 - (\frac{z_o}{z_F})^{1/2} + \frac{v_o}{w_F}ln(\frac{1-\frac{v_o}{w_F}}{(\frac{z_o}{z_F})^{1/2} - \frac{v_o}{w_F}})
\end{equation}

Where $z_o$ is the particle height, again assumed to be the crown of the tree. $z_F$ is the flame height. The height of the flame is found as the vertical height from the ground to the tip of the flame \cite{firebehaveref}. $v_o$ is the terminal velocity of the particle and $w_F$ is the flame gas velocity. Again, the data for flame gas velocity was not available to the researchers at this time. A workaround to this problem may be achieved by calculating the ratio of $v_o / w_F$ as:  
\begin{equation}
v_o/w_F = B(D_p/z_F)^{1/2}
\end{equation}

Where $D_p$ is the particle diameter, and $B$ is a constant set to 40. 

The second time section that needs to be calculated $t_2$ is the time it takes the particle to travel between the flame tip and the buoyant plume and is given by: 
\begin{equation}
t_2 = 0.2 + B(\frac{D_p}{z_F})^{1/2}(1 + B(\frac{D_p}{z_F})^{1/2}ln(1 + \frac{1}{1-(\frac{D_p}{z_F})^{1/2}}))
\end{equation}

and $t_3$ is defined as the time it takes the particle to complete its ascent through the buoyant plume. It is given by: 
\begin{equation}
t_3 = \frac{a_x}{0.8\frac{v_o}{w_F}}(ln(\frac{1 - 0.8 \frac{v_o}{w_F}}{1 - 0.8r \frac{v_o}{w_F}}) - 0.8\frac{v_o}{w_F}(r-1) - \frac{1}{2}(0.8\frac{v_o}{w_F})^2(r-1)^2)
\end{equation}

Where: 
\begin{flalign*}
r &= (\frac{(b_x + \frac{z_o}{z_F})}{a_x})^{1/2}\\
a_x &= 5.963\\
b_x &= 4.563
\end{flalign*}

$a_x$ and $b_x$ are constants defined by Albini \cite{albini} and relate to the flame structure. Albini defines the height to which a fire brand will travel as the point in time where the buoyant flow structure of the tree $t_f$ equals $t_t$. The buoyant flow time is defined as: 
\begin{equation}
t_f = t_o + 1.2 + \frac{a_x}{3}((\frac{b_x + \frac{z_o}{z_F}}{a_x})^{3/2} - 1)
\end{equation}
In this implementation, these values had to be approximated in several ways because not all of the necessary data was available. The maximum height was assumed to be constant, and while the equations are implemented and ready for the correct data, it is not yet available. 

A particle will be lofted into the air, and once it reaches its maximum height, the wind then carries it in a direction. If there is no wind in a simulation, the ember would simply fall back to the ground in the cell from which it was launched. Wind is modeled in this library as a 2D vector in each cell of the simulator. The vector contains a magnitude and direction of the wind. The wind vector only contains horizontal data. As in FARSITE's implementation, the fire brand's rate of descent decreases as it falls because as with a real brand, it loses density as it is in the air. Examples of fire brand sizes and their trajectories may be seen in Figure \ref{fig:spot_diagram}. The elevation at any time $t$ is given by: 
\begin{equation}
z(t) = z(0) - v_o(0)(\frac{t}{\tau} - \frac{1}{2}(\frac{t}{\tau})^2)
\end{equation}

Where: 
\begin{flalign*}
\tau &= \frac{4C_dv_o(0)}{K\pi g} \\
K &= 0.0064\\
g &= 9.8 m s^{-2}\\
v_o &= (\frac{\pi g \rho_s D_p}{2C_d\rho_a})^{1/2}\\
\rho_s &= 0.3 g cm^{-3}\\
\rho_a &= 1.2 x 10^{-3} g cm^{-3}\\
C_D &= 1.2
\end{flalign*}

Where $g$ is the gravitational constant, $\rho_s$ is the density of charred wood cylinder, $\rho_a$ is the density of air, and $D_d$ is the drag coefficient of cylindrical particles. 

As the particle descends along its trajectory, its rate of speed in the direction $X$ is determined by the windspeed vector in the cell in which the brand is currently falling. The rate of descent decreases logarithmically as it approaches the top of the tree canopies. 
\begin{equation}
\frac{dX}{dt} = U_H\frac{ln(\frac{z}{z_f})}{ln(\frac{H}{z_f})}
\end{equation}

Where $z$ is the current height of the particle, $H$ is the current height of the forest canopy, and $z_f$ is the friction length of the particle, which is set to $0.4306H$. In this implementation of the paper, there is no modeling of 3-dimensional wind, which means that the wind vector does not change based on height. However, in the interest of presenting a complete model for future work, the windspeed at any height $H$ is given as: 
\begin{equation}
U_H = \frac{U_{20+H}}{ln(\frac{20+1.18H}{0.43H})}
\end{equation}

In this paper, the method for ignition is modeled by a simple probability which is input by the user. However, there are more complex methods for modeling the ignition of a firebrand which impacts the forest floor. Factors such as forest floor fuel type, forest canopy filtering, surface moisture content, and the temperature of the fuel \cite{bradshaw1984,blackmarr1972,bunting1974}. These factors are not examined and are left to future work. Once a fire ignites, it is treated the same as any other ignition source. 

In this paper, spotting introduced new challenges to the development of the library. First, it required new data which was not available to the project. Second, the fire simulations, as seen in the previous sections, propagate the fire by testing to see if a cell has already reached the point of ignition. This point of ignition is determined by the value of the time of arrival for the cell being above zero (which represents the initial fire), and below the value which is set as infinity. In practice this value is the maximum value possibly held by a data type. The problem introduced by spotting is that the time of arrival that is calculated by the spotting kernel is a time of arrival that will occur in the future. As the simulation code was written for the previous sections, the fire would begin to propagate the exact next time tick. To fix this issue, the descent of the falling particle is computed at each time tick with the rest of the simulations and stored in a list of falling embers. The list is updated when one burns out or lands to ignite a new fire, or a new ember needs to be added to the list.  

If the spotting brand's ignition occurs after a cell has already been lit, then it is ignored. If the brand arrives at a time when a cell is not lit, it lights a fire in that cell, which propagates as a normal fire from that point would. 

\section{Simulation Progression}
The simulation has the ability to adapt to the needs of the user. Runtime analysis may be found in Chaper \ref{chapter:results}, which gives an illustration of the impact of adding more features to a simulation. The more features there are to calculate, the longer the runtime will be. However, it is up to the user of the library to decide what is appropriate for their application. The resolution of the fire and desired propagation features allow the user to use the library to create a simulation capable of a flow much like what is seen again in Algorithm \ref{alg:sim2}.

\begin{algorithm}
  \caption{Simulation Progression}
  \label{alg:sim2}
%   \begin{algorithmic}[1] <- this will give you line numb
  \begin{algorithmic}
  \STATE InitializeTerrainData();
  \STATE CalculateSpreadRates();
  \WHILE{Simulation !Complete}
  \STATE RunPropagationSimulation();
  \ENDWHILE  
  \STATE GenerateOutputFile();
  \end{algorithmic}
\end{algorithm}

The specifics of the library's use are left up to the needs of the user. Results for simulation tests may be seen in Chapter \ref{chapter:results}. 