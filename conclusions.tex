\chapter{Conclusions and Future Work}
\label{chapter:conclusions}

\section{Conclusions}
This work implemented vFireLib: a fully-functional forest fire simulation library. This library has the ability to simulate three propagation methods that include fire acceleration, crowning, and spotting. The simulation library provides a simple interface to initialize the simulation, provides a method for easily interfacing with the graphics processing unit, runs a simulation, and outputs a time of arrival map. 

\section{Future Work}
While this work provides a fully-functioning forest fire library, there are several areas where it could be expanded in the future. 

\subsection{Improved Spotting} 
Several approximations were made when implementing the spotting method for this work, as presented in Chapter \ref{chapter:implementation}. The limitations were mostly based in data limitations rather than simulation limitations. However, a more accurate method for describing the spotting needs additional data. A more comprehensive model of the atmosphere above a forest fire would increase the accuracy of the spotting method as it would allow for multiple layers of wind vectors to show the impact of wind on the fire itself. The implementation of the spotting calculates a "future burn" when a fire brand ignites a fire ahead of the fire. This map of future burns can be used to queue additional desired fires by the user, which could allow fire scientists more detailed control over the simulation. 

Additional work needs to be done on the method for calculating whether a fire brand ignites a new fire at it's point of contact with the forest floor. The current system only uses a user-defined probability, but there are more factors which influence it, such as the type of fuel found on the forest floor, the filtering of the forest canopy, surface moisture content, and the temperature of the fuel \cite{bradshaw1984,blackmarr1972,bunting1974}.

\subsection{Atmospheric Incorporation}
Forest researchers are interested in the carbon footprint caused by a forest fire. This environmental cost is just as important as the physical cost of loss of property and life that can be caused by a wildfire. As a fire burns, it releases a substantial amount of carbon dioxide into the atmosphere. Knowing the impact of a fire could encourage prevention methods or understanding of a previously existing fire on the environment. 

With the increase in need for understanding atmospheric impact of a forest fire, a better model of the interaction between the fire and wind data is needed. A wind mesh could increase the accuracy of the spotting method, as mentioned previously, but could also be used to determine the direction the smoke will be blown as well.

\subsection{Multi-GPU Implementation} 
CUDA provides the ability to program for multiple GPUs at the same time \cite{cuda}. The ability to use multiple GPUs to process the simulation could further improve the runtime. Problems that will arise from implementing the library on multiple GPUs include stitching the simulation together at the edges and triggering memory transfer between GPUs at the point where simulations bleed across the cells being processed on individual GPUs. 

\subsection{Visualization}
In order to better understand the impacts of forest fires in their environment, a visualization is needed. Visualizing the simulation is an important next step for future work of this project. It would allow an interactive environment in which the simulation could exist and be modified dynamically. 

The fire library is built to be used as a tool to run calculations for any program which wished to utilize it. The library should allow for dynamic changes to the terrain values which could simulate real-world firefighting tools such as bulldozing tree lines, water and fire retardant drops, as well as tests to evaluate the effectiveness of fire containment methods by the forest service. 